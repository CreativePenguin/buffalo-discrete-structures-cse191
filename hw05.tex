\documentclass[a4paper]{article} 
\usepackage[inline]{enumitem}
\input{head}
\begin{document}

%-------------------------------
%	TITLE SECTION
%-------------------------------

\fancyhead[C]{}
\hrule \medskip % Upper rule
\begin{minipage}{0.295\textwidth} 
\raggedright
\footnotesize
Winston Peng \hfill\\   
50364686 \hfill\\
wpeng2@buffalo.edu
\end{minipage}
\begin{minipage}{0.4\textwidth} 
\centering 
\large 
Homework 5\\ 
\normalsize 
Intro To Discrete Structures, 2021\\ 
\end{minipage}
\begin{minipage}{0.295\textwidth} 
\raggedleft
\today\hfill\\
\end{minipage}
\medskip\hrule 
\bigskip

\section{Recurrence Relations (4 x 2 pts)}
\begin{enumerate}
    \item \textbf{A person deposits \$5000 in a bank account that yields 7\% interest compounded annually.}
    \begin{enumerate}
        \item \textbf{Set up a recurrence relation for the amount in the account at the end of \textit{n} years.}\\
        \boldmath $a_0 = $ \unboldmath $5000$\\
        \boldmath $a_n = $ \unboldmath $a_{n-1} * 1.07$
        \item \textbf{How much money will the account contain after 7 years? Show your calculation.}\\
        $a_0 = 5000$\\
        $a_1 = a_0 * 1.07 = 5000 * 1.07 = 5350$ \\
        $a_2 = a_1 * 1.07 = 5350 * 1.07 = 5724.5$ \\
        $a_3 = a_2 * 1.07 = 5724.5 * 1.07 = 6125.22$ \\
        $a_4 = a_3 * 1.07 = 6125.22 * 1.07 = 6553.99$ \\
        $a_5 = a_4 * 1.07 = 6553.99 * 1.07 = 7012.77$ \\
        $a_6 = a_5 * 1.07 = 7012.77 * 1.07 = 7503.66$ \\
        $a_7 = a_6 * 1.07 = 7503.66 * 1.07 = 8028.92$
    \end{enumerate}
    \item \textbf{Suppose that the number of bacteria in a colony doubles every hour.}
    \begin{enumerate}
        \item \textbf{Set up a recurrence relation for the number of bacteria after \textit{n} hours have elapsed.}\\
        \boldmath $a_n = $ \unboldmath $a_{n - 1} * 2$
        \item \textbf{If 150 bacteria are used to begin a new colony, how many bacteria will be in the colony in 6 hours? Show your calculation.}\\
        $a_0 = 150$\\
        $a_1 = a_0 * 2 = 150 * 2 = 300$\\
        $a_2 = a_1 * 2 = 300 * 2 = 600$\\
        $a_3 = a_2 * 2 = 600 * 2 = 1200$\\
        $a_4 = a_3 * 2 = 1200 * 2 = 2400$\\
        $a_5 = a_4 * 2 = 2400 * 2 = 4800$\\
        $a_6 = a_5 * 2 = 4800 * 2 = 9600$\\
    \end{enumerate}
\end{enumerate}

\bigskip %--------------------------

\section{Mathematical Induction (10 pts)}
\boldmath {\large \textbf{Prove for every positive integer \textit{n}, $1^3 + 2^3 + 3^3 + ... + n^3 = \frac{n^2(n+1)^2}{4}$ is true}} \unboldmath \\
P(n): $\sum_{i=1}^n i^3 = \frac{n^2(n+1)^2}{4}$\\

\textbf{\ul{Basis Step:}}\\
Goal: P(1) is true.\\
Case n = 1: $\sum_{i=1}^1 i^3 = 1$\\
$\frac{n^2(n+1)^2}{4} = \frac{1^2(1+1)^2}{4} = 1$\\
Therefore, P(1) is true.\\

\textbf{\ul{Inductive Step:}}\\
Goal: $\forall k \ge 1$, (P$(k) \implies $P$(k+1))$ is true. Let $k \ge 1$ \\

\ul{\textbf{Inductive Hypothesis:}} 
Assume that P(\textit{k}) is true for an arbitrary integer \textit{k} with $k \ge 1$
\begin{center} 
$P(k): \sum_{i=1}^k i^3 = 1^3 + 2^3 + ... + k^3 = \frac{k^2(k+1)^2}{4}$
\end{center}
    Under this assumption, we must show that p(k + 1) is true.
\begin{center}
    $P(k + 1): \sum_{i=1}^{k+1} i^3 = \frac{(k + 1)^2(k + 2)^2}{4}$
\end{center}
$\sum_{i=1}^{k+1} i^3 = \sum_{i=1}^{k} i^3 + (k + 1)^3 = 
\frac{k^2(k + 1)^2}{4} + (k + 1)^3 = \frac{(k + 1)^2(k + 2)^2}{4}$\\
This shows that $P(k + 1)$ is true when $P(k)$ is true, therefore by principle of mathematical induction, for all $n \ge 1$, $P(n)$ is true.\\

% \frac{k^2(k + 1)^2 + 4(k + 1)^3}{4} = 
% \frac{k^2(2k^2 + 2k + 1) + 4(k^3 + 3k^2 + 3k + 1^3)}{4} = 
% \frac{2k^4 + k^22k + k^2 + 4k^3 + 12k^2 + 12k + 4}{4}$\\

\bigskip %------------------------------------------

\section{Strong Induction (10 pts)}
\ul{\textbf{Given:}} $a_0 = 0,\: a_1 = 4,\: a_n = 6a_{n - 1} - 5a_{n - 2}$ for $n \ge 2$\\
\ul{\textbf{Prove:}} $P(n): a_n = 5^n - 1$ for all $n \ge 0$\\

\ul{\textbf{Basis Step:}}\\
$P(0): a_0 = 0$,\quad $5^0 - 1 = 0$
$P(1): a_1 = 4$,\quad $5^1 - 1 = 4$
$P(0)$ and $P(1)$ are true.\\

\ul{\textbf{Inductive Step:}}\\
We must show that if P(i) is true for $0 \le i \le k$, where $k \ge 1$, then $P(k + 1)$ is true.\\

\ul{\textbf{Inductive Hypothesis:}} Let $k \ge 1$.\\
Assume $k \ge 1$ and P(i) is true for $0 \le i \le k$, where $k \le 1$ are true.
\begin{center}
    $P(k): a_k = 6a_{k - 1} - 5a_{k - 2}$
\end{center}
Assuming P(k), we must show P(k + 1).
\begin{center}
    $P(k + 1): a_{k + 1} = 5^{k + 1} - 1$
\end{center}
$a_{k + 1} = 6a_{k} - 5a_{k - 1}$, By recurrence relation\\
$a_{k + 1} = 6(5^k - 1) - 5(5^{k - 1} - 1)$, By inductive hypothesis\\
$a_{k + 1} = 5^{k + 1} - 1$\\
Therefore, $P(k + 1)$ is true when $P(i)$ is true for $0 \le i k$, where $k \le 1$, and the proof of the inductive step is complete.
% $a_{k + 1} = 6(6a_{k - 1} - 5a_{k - 2}) - 5(6a_{k - 2} - 5a_{k - 3}) =$

\section{Combinations and Permutations (10pts + 2pts Extra Credit)}
\begin{enumerate}
    \item \textbf{You want to make 4-letter codes using only the uppercase English letters. How many of these codes have no two consecutive letters the same?}\\
    $26 * 25 * 25 * 25 = 406250$
    \item \textbf{Consider the following sets of characters that you can use to make a password.}
    
    \textbf{Letters = \{b, q, s v, x\}} \\
    \textbf{Digits = \{0, 1, 2, 3, 4, 5, 6, 7, 8, 9\}}\\
    \textbf{Specials = \{\#, \&, @, \$, *\}}
    
    \textbf{The passwords can be of length 6, 7, or 8. The first character cannot be a digit. How many passwords are there?}\\
    6 digit password: $10 * 20 * 20 * 20 * 20 * 20 = 32000000$\\
    7 digit password: $10 * 20 * 20 * 20 * 20 * 20 * 20 = 640000000$\\
    8 digit password: $10 * 20 * 20 * 20 * 20 * 20 * 20 * 20 = 12800000000$\\
    Total: 13472000000
    \item \textbf{Phone numbers are 7 digits long and start with either 716 or 718. How many different phone numbers are there in which the last four digits are all different?}\\
    $2 * 10 * 9 * 8 * 7 = 10080$
    \item \textbf{Bob has 5 hw assignments but can only do 3. How many ways can he choose 3?}\\
    $5 * 4 * 3 = 60$
    \item \textbf{Show that if 6 integers are chosen at random, at least two nums will have the same remainder when divided by 5. Use pigeonhole principle.}\\
    There are 6 integers, 6 - 1 = 5. Therefore, it stands to reason that two numbers that will have the same remainder when divided by five, as there are only 5 possible remainders of 5.
    \item \textbf{How many unordered selections of three elements can be made from the set \{a, b, c, d, e, f\}?}\\
    $6^3 = 216$
\end{enumerate}

\bigskip %-----------------------------

\section{Isomorphic (6pts = 3 x 2)}
\begin{enumerate}
    \item The graphs are isomorphic as the number of edges and vertices for the two graphs are identical.
    \item The graphs are not isomorphic as the vertices do not have the same degree between the two graphs.
\end{enumerate}

\bigskip %------------------------------

\section{Graphs (10pts = 2 x 5(+ 4 points extra)}
\begin{enumerate}
    \item Walk
    \item Walk, Circuit, Cycle
    \item Walk, Trail
    \item Walk
    \item Walk, Trail, Path
    \item Not walk
    \item Walk, Circuit
\end{enumerate}
\end{document}
