\documentclass[a4paper]{article} 
\usepackage[inline]{enumitem}
\usepackage{tasks}
\usepackage{array}
\usepackage{tabularx}
\settasks{label =•}
\usepackage{enumitem}
\setenumerate{label = \alph*)}
\newcolumntype{Y}[1]{>{\centering\arraybackslash} X{#1}}
\addtolength{\hoffset}{-2.25cm}
\addtolength{\textwidth}{4.5cm}
\addtolength{\voffset}{-3.25cm}
\addtolength{\textheight}{5cm}
\setlength{\parskip}{0pt}
\setlength{\parindent}{0in}

%----------------------------------------------------------------------------------------
%	PACKAGES AND OTHER DOCUMENT CONFIGURATIONS
%----------------------------------------------------------------------------------------

\usepackage{blindtext} % Package to generate dummy text
\usepackage{charter} % Use the Charter font
\usepackage[utf8]{inputenc} % Use UTF-8 encoding
\usepackage{microtype} % Slightly tweak font spacing for aesthetics
\usepackage[english, ngerman]{babel} % Language hyphenation and typographical rules
\usepackage{amsthm, amsmath, amssymb} % Mathematical typesetting
\usepackage{float} % Improved interface for floating objects
\usepackage[final, colorlinks = true, 
            linkcolor = black, 
            citecolor = black]{hyperref} % For hyperlinks in the PDF
\usepackage{graphicx, multicol} % Enhanced support for graphics
\usepackage{xcolor} % Driver-independent color extensions
\usepackage{marvosym, wasysym} % More symbols
\usepackage{rotating} % Rotation tools
\usepackage{censor} % Facilities for controlling restricted text
\usepackage{listings, style/lstlisting} % Environment for non-formatted code, !uses style file!
\usepackage{pseudocode} % Environment for specifying algorithms in a natural way
\usepackage{style/avm} % Environment for f-structures, !uses style file!
\usepackage{booktabs} % Enhances quality of tables
\usepackage{tikz-qtree} % Easy tree drawing tool
\tikzset{every tree node/.style={align=center,anchor=north},
         level distance=2cm} % Configuration for q-trees
\usepackage{style/btree} % Configuration for b-trees and b+-trees, !uses style file!
\usepackage[backend=biber,style=numeric,
            sorting=nyt]{biblatex} % Complete reimplementation of bibliographic facilities
\addbibresource{ecl.bib}
\usepackage{csquotes} % Context sensitive quotation facilities
\usepackage[yyyymmdd]{datetime} % Uses YEAR-MONTH-DAY format for dates
\renewcommand{\dateseparator}{-} % Sets dateseparator to '-'
\usepackage{fancyhdr} % Headers and footers
\pagestyle{fancy} % All pages have headers and footers
\fancyhead{}\renewcommand{\headrulewidth}{0pt} % Blank out the default header
\fancyfoot[L]{} % Custom footer text
\fancyfoot[C]{} % Custom footer text
\fancyfoot[R]{\thepage} % Custom footer text
\newcommand{\note}[1]{\marginpar{\scriptsize \textcolor{red}{#1}}} % Enables comments in red on margin
\usepackage{array}
\usepackage{tabularx}
\newcolumntype{Y}[1]{>{\centering\arraybackslash} X{#1}}

% \usepackage{enumerate}

% Change default enumerate to use letters and roman numerals
\usepackage{enumitem}
\setenumerate[1]{label = \alph*)}
\setenumerate[2]{label = \roman*)}

% Fix latex underline (use \ul)
\usepackage{soul}
\setuldepth{WordWithNoDescenders}
% \setuldepth{pqgy}

%----------------------------------------------------------------------------------------

\begin{document}

%-------------------------------
%	TITLE SECTION
%-------------------------------

\fancyhead[C]{}
\hrule \medskip % Upper rule
\begin{minipage}{0.295\textwidth} 
\raggedright
\footnotesize
Winston Peng \hfill\\   
50364686 \hfill\\
wpeng2@buffalo.edu
\end{minipage}
\begin{minipage}{0.4\textwidth} 
\centering 
\large 
Homework 1\\ 
\normalsize 
Intro To Discrete Structures, 2021\\ 
\end{minipage}
\begin{minipage}{0.295\textwidth} 
\raggedleft
\today\hfill\\
\end{minipage}
\medskip\hrule 
\bigskip

%-------------------------------
%	CONTENTS
%-------------------------------

\section{Is or Is Not Proposition?}
%\blindtext
%\subsection{}
%Some equations
\begin{enumerate}
    \item\textbf{CSE 115 is a prerequisite of CSE 191.}
    
    Proposition. CSE 115 is not a prerequisite of CSE 191.
    
    \item\textbf{Do we have class today?}
    
    Not a proposition. Questions are not declarative statements.
    
    \item\textbf{Amazon is a rainforest.}
    
    Proposition. Amazon is not a rainforest.
    
    \item\textbf{All integers are even.}
    
    Proposition. Not all integers are even.
    
    \item\textbf{Give me a glass of water.}
    
    Not a proposition. Commands are not declarative statements.
\end{enumerate}
% \begin{align*}
% y &=  \sum\limits_{i,k} m_i \cdot f^k \\
% x &=  
% \underset{11}{\underbrace{3 + 8}} + 5 + 7
% \end{align*}

% \subsection{Second Subtask}
% \blindtext

\bigskip

%------------------------------------------------

\section{Evaluate Truth Value}
% \begin{itemize*}
%     \item p: False (F).
%     \item q: True (T).
% \end{itemize*}
\begin{tasks}(4)
    \task \textit{p}:False(F).
    \task \textit{q}:True(T)
    \task \textit{r}:False(F).
    \task \textit{s}:True(T).
\end{tasks}
\begin{enumerate}
    \item \boldmath $p \implies q \land r$ \unboldmath
    
    $F \implies T \land F$\\
    $F \implies F$\\
    $T$
    \item \boldmath $(p \implies q) \land r$ \unboldmath
    
    $(F \implies T) \land F$\\
    $T \land F$\\
    $T$
    \item \boldmath $p \land \neg q \iff q \implies p$ \unboldmath
    
    $F \land \neg T \iff T \implies F$\\
    $F \iff F$\\
    $T$
    \item \boldmath $p \land \neg q \oplus r \implies \neg s \lor q$ \unboldmath
    
    $F \land \neg T \oplus F \implies \neg T \lor T$\\
    $F \oplus F \implies T$\\
    $F \implies T$\\
    $T$
    \item \boldmath $(p \land \neg q) \oplus (r \implies (\neg s \lor q))$ \unboldmath
    
    $(F \land \neg T) \oplus (F \implies (\neg T \lor T))$\\
    $F \oplus (F \implies T)$\\
    $F \oplus T$\\
    $T$
\end{enumerate}
% \subsection{First Subtask}

\bigskip

%------------------------------------------------

\section{Truth Tables}
% \begin{boldmath} 32 * 15 + p \implies 13 \end{boldmath} \\
% $32 * 15 + p \implies 13$\\
\begin{enumerate}
    \item Make truth table for $p \lor (\neg q \land r)$
    
    \begin{center}
        % \begin{tabular}{ | c | c | c |c| }
        % \begin{tabular}{ | m{1cm} | m{1cm} | m{1cm} |m{5cm}| }
        \begin{tabularx}{0.8\textwidth}{ | >{\centering\arraybackslash} X | >{\centering\arraybackslash} X | >{\centering\arraybackslash} X | >{\centering\arraybackslash} X | }
        % \begin{tabularx}{0.8\textwidth}{ | X | X | X | X | }
        \hline
        % \begin{centering}
        \textbf{p} & \textbf{q} & \textbf{r} & $\mathbf{p \boldsymbol{\lor} \boldsymbol{(\neg} q \boldsymbol{\land} r\boldsymbol{)}}$ \\ [1ex]
        % {\boldmath p} & {\boldmath q} & {\boldmath r} & {\boldmath p \lor (\neg q \land r)} 
        % \bold
        % \end{centering}
        \hline
        T & T & T & T \\
        \hline
        T & T & F & T \\
        \hline 
        T & F & T & T \\
        \hline
        T & F & F & T \\
        \hline
        F & T & T & F \\
        \hline
        F & T & F & F \\
        \hline
        F & F & T & T \\
        \hline
        F & F & F & F \\
        \hline
        \end{tabularx}
    \end{center}
    \item Find correct logical expression for truth table
    
    $\neg p \lor \neg q$
    
    \begin{center}
        \begin{tabularx}{0.8\textwidth}{| Y | Y | Y |}
        \hline
        \textbf{p} & \textbf{q} & \textbf{?} \\ [1ex]
        \hline
        F & F & T \\
        \hline
        F & T & T \\
        \hline
        T & F & T \\
        \hline
        T & T & F \\
        \hline
        \end{tabularx}
    \end{center}
\end{enumerate}

\bigskip

%-------------------------------------------------------

\section{Translate Math To English}
\textbf{Propositions:}
\begin{itemize}
    \item p: It was below freezing
    \item q: it was snowing
    \item r: the road was closed
\end{itemize}
\textbf{Problems:}
\begin{enumerate}
\boldmath
    \item $q \boldsymbol{\implies} p$ If it was snowing, then the road was closed. \\
    \item $p \land q$ It was below freezing and it was snowing. \\
    \item $\neg(p \land r)$ It was not below freezing and the road was not closed. \\
    \item \textbf{$r \implies (p \lor q)$} If the road was closed, it was either below freezing or snowing. \\
    \item \textbf{$r \iff (p \land q)$} The road was closed if and only if it was below freezing and snowing. \\
\unboldmath
\end{enumerate}

\bigskip

%-----------------------------------------------------------

\section{Translate English To Math}
\textbf{Propositions:}
\begin{itemize}
    \item p: The application has passed the learner permit test.
    \item q: The applicant has passed the road test.
    \item r: The applicant is allowed a driver's licence.
\end{itemize}
\textbf{Problems:}
\begin{enumerate}\bfseries
    \item The applicant did not pass the road test but passed the learner permit test. $\neg q \land p$ \\
    \item If the applicant passes the learner permit test and the road test, then the applicant is allowed a driver's license. $(p \land q) \implies r$ \\
    \item Passing the learner permit test and the road test are necessary for being allowed a driver's license. $(p \land q) \implies r$ \\
    \item The applicant passed either the learner permit test or the road test, but not both $p \oplus q$ \\
    \item It is not true that the applicant does not pass the road test and is allowed a driver's license. $\neg(\neg q \implies r)$
\end{enumerate}

\bigskip

%---------------------------------------------------------

\section{Tautology, Contingency, or Contradiction?}
\textbf{What is \boldmath $(p \land q) \implies q$}? \unboldmath\\
\begin{center}
\begin{tabularx}{0.8\textwidth}{| Y | Y | Y |}
    \hline
    \textbf{\textit{p}} & \textbf{\textit{q}} & \boldmath $(p \land q) \implies q$ \unboldmath \\ [1ex]
    \hline
    T & T & T \\
    \hline
    T & F & T \\
    \hline
    F & T & T \\
    \hline
    F & F & T \\
    \hline
\end{tabularx}
\end{center}
$(p \land q) \implies q$ is a tautology, as it always results in T. \\

\bigskip 

%-----------------------------------------------------------

\section{Extra Credit: Equivalence Laws}
Show $\neg(\neg p \land q) \land (p \lor q)$ is logically equivalent to $p$. \\
\begin{tabularx}{0.4\textwidth}{X X}
$\neg(\neg p \land q) \land (p \lor q)$ & Hypothesis \\
$(p \lor \neg q) \land (p \lor q)$ & De Morgan's Law \\
$p \land p$ & Absorption Law \\
$p$ & Idempotent Law
% $(\neg q \land p) \land (p \lor q)$ & Commutative Law \\
% $\neg q \land (p \land p) \lor q$ & Associative Law \\
% $\neg q \land p \lor q$ & Idempotent Law \\
% $\neg q \lor q \land p$ & Commutative Law \\
% $T \land p$ & Negation Law \\
% $p$ & Identity Law \\
\end{tabularx}

\bigskip

Show $\neg(p \lor \neg(p \land q))$ is logically equivalent to $F$. \\
\begin{tabularx}{0.4\textwidth}{X X}
$\neg(p \lor \neg(p \land q))$ & Hypothesis \\
$\neg(p \lor (\neg p \lor \neg q))$ & De Morgan's Law \\
$\neg p \land (p \land q)$ & De Morgan's Law \\
$(\neg p \land p) \land q$ & Associative Law \\
$F \land q$ & Negation Law \\
$F$ & Domination Law
% $(\neg p \land p) \land (\neg p \land q)$ & Distributive Law \\
% $F \land (\neg p \land q)$ & Negation Law \\
% $(F \land \neg p) \land (F \land q)$ & Distributive Law \\
% $F \land F$ & Domination Law \\
% $F$
% $\neg p \lor q$ & Identity Laws \\
% $p \implies q$ & Conditional Identity
\end{tabularx}

\end{document}
