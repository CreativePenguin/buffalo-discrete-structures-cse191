\documentclass[a4paper]{article} 
\usepackage[inline]{enumitem}
\input{head}
\begin{document}

%-------------------------------
%	TITLE SECTION
%-------------------------------

\fancyhead[C]{}
\hrule \medskip % Upper rule
\begin{minipage}{0.295\textwidth} 
\raggedright
\footnotesize
Winston Peng \hfill\\   
50364686 \hfill\\
wpeng2@buffalo.edu
\end{minipage}
\begin{minipage}{0.4\textwidth} 
\centering 
\large 
Homework 3 - Sets\\ 
\normalsize 
Intro To Discrete Structures, 2021\\ 
\end{minipage}
\begin{minipage}{0.295\textwidth} 
\raggedleft
\today\hfill\\
\end{minipage}
\medskip\hrule 
\bigskip

\section{Translate English To Math - Sets}
\large Sets: \smallskip \\
\normalsize
A: The set of students who are taking cs majors \\
B: The set of students taking CSE 191

\begin{enumerate}
    \item \textbf{The set of cs majors taking CSE 191} $A \cap B$
    \item \textbf{The set of cs majors not taking CSE 191} $A \cap \neg B$
    \item \textbf{The set of either cs majors or CSE 191 students} $A \cup B$
    \item \textbf{The set of students either not cs majors or not taking CSE 191} $\neg A \cup \neg B$
\end{enumerate}

\bigskip %-------------------------

\section{Generalized Set Operators}
Let $A_i = \{-i, ...,-2, -1, 0, 1, 2, ..., i\}$
\begin{enumerate}
    \item \boldmath $\bigcup\limits_{i=1}^n A_i$ \unboldmath $= A_1 \cup A_2 \cup ... \cup A_n = \{-1, 0, 1\} \cup \{-2, -1, 0, 1, 2\} \cup ... \cup \{-n, ..., -1, 0, 1, ..., n\}$ \\
    \boldmath $A$\unboldmath $_n$ \boldmath $ = \{$ \unboldmath 
    $-n, ..., -1, 0, 1, ..., n$ 
    \boldmath $\}$
    \item $\bigcap\limits_{i=1}^n A_i$ \unboldmath $= A_1 \cap A_2 \cap ... \cap A_n = \{-1, 0, 1\} \cap \{-2, -1, 0, 1, 2\} \cap ... \cap \{-n, ..., -1, 0, 1, ..., n\}$ \\
    \boldmath $A$\unboldmath $_1$ \boldmath $= \{$ \unboldmath $-1, 0, 1$ \boldmath $\}$
\end{enumerate}

\section{Set Operations}
Describe resulting set using roster notation \\
% \setlength\parindent{24pt} 
\hspace*{10mm} A = $(\{0\} \cup \{x \, | \, x \in \mathbb{Z}^+$ and $x < 4\})$ \\
\hspace*{10mm} B = $\{0, 1, 3, 5, 7\}, and$ \\
\hspace*{10mm} C = $(\{0\} \cup \{x \, | \, x \in \mathbb{Z}^+$ and $x$ is even and $x < 8\})$

\begin{enumerate}
    \item \boldmath $A \oplus B$ \unboldmath \\
    $\{2, 5, 7\}$
    \item \boldmath $(A \cup B \cup C) - (A \cap B \cap C)$ \unboldmath \\
    $\{1, 2, 3, 4, 5, 6, 7\}$
    \item \boldmath $B \oplus (A \cap C)\\
    $ \unboldmath $\{2\}$
    \item \boldmath $B \oplus B$ \unboldmath \\
    $\{\}$
    \item \boldmath $C \oplus \varphi$ \unboldmath \\
    $\{0, 2, 4, 6\}$
\end{enumerate}

\bigskip %-----------------------------

\section{Set Operations}
Describe (a-e) using roster notation \\
\hspace*{10mm} A = $\{0, 1\}$, \\
\hspace*{10mm} B = $\{x\}$, and \\
\hspace*{10mm} C = $\{red, green, blue\}$. \\

\begin{enumerate}
    \item \textbf{(A $\times$ B) $\times$ C} \\
    \{(0, x), (1, x), ((0, x), red), ((1, x), red), ((0, x), green), ((1, x), green), ((0, x), blue), ((1, x), blue)\}
    \item \textbf{$\mathcal{P}$(A $\times$ B)} \\
    $\mathcal{P}$\{(0, x), (1, x)\} \\
    \{$\O$, (0, x), (1, x), \{(0, x), (1, x)\}\}
    \item \textbf{$\mathcal{P}$(A) $\times \mathcal{P}$(B)} \\
    \{$\O$, \{0\}, \{1\}, \{1, 2\}\} $\times$ \{$\O$, \{x\}\} \\
    \{($\O$, $\O$), ($\O$, \{x\}), (\{0\}, $\O$), (\{0\}, \{x\}), (\{1\}, $\O$), (\{1\}, \{x\}), (\{1, 2\}, $\O$), (\{1, 2\}, \{x\})\}
    \item \textbf{A$^0 \cup $ A$^2 \cup$ A$^3$} \\
    \{()\} $\cup$ \{0, 1\} $\times$ \{0, 1\} $\cup$ \{0, 1\} $\times$ \{0, 1\} $\times$ \{0, 1\} \\
    \{()\} $\cup$ \{(0, 0), (1, 0), (0, 1), (1, 1)\} $\cup$ \{(0, 0, 0), (0, 0, 1), (0, 1, 0), (0, 1, 1), (1, 0, 0), (1, 0, 1), (1, 1, 0), (1, 1, 1)\} \\
    \{(), (0, 0), (1, 0), (0, 1), (1, 1), (0, 0, 0), (0, 0, 1), (0, 1, 0), (0, 1, 1), (1, 0, 0), (1, 0, 1), (1, 1, 0), (1, 1, 1)\}
    \item \textbf{B$^2 \cap$ B$^4$} \\
    \{x\} $\times$ \{x\} $\cap$ \{x\} $\times$ \{x\} $\times$ \{x\} $\times$ \{x\} \\
    \{(x, x)\} $\cap$ \{(x, x, x ,x)\} \\
    \{$\O$\}
    \item \textbf{$|$C$^5|$} \\
    $2^5$\\
    $32$
\end{enumerate}

\section{Extra Credit - Evaluate Equivalence}
\begin{enumerate}
    \item \textbf{If (A $\cup$ C) = (B $\cup$ C), then A = B.}\\
    This statement is false. If A is \{1, 2, 3\} and C is \{3, 4, 5\}, then B could either be \{1, 2\} or \{1, 2, 3\}. Since duplicates don't matter in sets, both values of B would be equivalent, meaning that A doesn't always equal B.
    \item \textbf{If (A $\cap$ C) = (B $\cap$ C), then A = B.}\\
    This statement is false. If A is \{1, 2, 3\} and C is \{3, 4, 5\}, then B could be \{2, 3\}, \{3\}, or \{1, 2, 3\}. As long as set B in the example has 3, and not 4 or 5, the statement would ring true. This means that A wouldn't always equal B.
\end{enumerate}

\end{document}
